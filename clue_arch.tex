%\tikzstyle{startstop} = [rectangle, rounded corners, 
%minimum width=2cm, 
%minimum height=1cm,
%text centered, 
%draw=black, 
%%fill=red!30
%]

\tikzstyle{io} = [trapezium, 
trapezium stretches=true, % A later addition
trapezium left angle=70, 
trapezium right angle=110, 
minimum width=2cm, 
minimum height=1cm, text centered, 
text width=2cm, 
draw=black, fill=blue!30]

\tikzstyle{startstop} = [rectangle, rounded corners, 
minimum width=2cm, 
minimum height=1cm, 
text centered, 
text width=2cm, 
draw=black, 
%fill=orange!30
]

\tikzstyle{decision} = [diamond, 
minimum width=3cm, 
minimum height=1cm, 
text centered, 
draw=black, 
fill=green!30]
\tikzstyle{arrow} = [thick,->,>=stealth]


\begin{figure}
\begin{tikzpicture}[node distance=3cm]

\node (dbs) [io] {OdeBase BiomodelsDB};
\node (io) [startstop,below of=dbs,yshift=1cm] {Model I/O};
\node (parse) [startstop, right of=io] {Parser};
\node (local) [io, above of=parse,yshift=-1cm] {Local model};
\node (clue) [startstop,right of=parse] {CLUE};
\node (linalg) [startstop, below of=clue,yshift=1cm] {Linear Algebra};
\node (sim) [startstop,right of= clue]{Simulation};

%\node (pro2b) [process, right of=dec1, xshift=2cm] {Process 2b};
%\node (out1) [io, below of=pro2a] {Output};
%\node (stop) [startstop, below of=out1] {Stop};

\draw [arrow] (dbs) -- (io);
\draw [arrow] (io) -- (parse);
\draw [arrow] (parse) -- node[anchor=north] {(2)} (clue);
\draw [arrow] (local) -- node[anchor=east] {(1)} (parse);
\draw [arrow] (linalg) -- (clue);
\draw [arrow] (clue) -- (sim);
%\draw [arrow] (dec1) -- node[anchor=east] {yes} (pro2a);
%\draw [arrow] (dec1) -- node[anchor=south] {no} (pro2b);
%\draw [arrow] (pro2b) |- (pro1);
%\draw [arrow] (pro2a) -- (out1);
%\draw [arrow] (out1) -- (stop);

\end{tikzpicture}
\caption{Architecture of \ToolName. 
    The Model I/O module interfaces with model repositories such as OdeBase and BiomodelsDB. 
    The Parsing module can take a model from the mentioned databases or a local model. 
    Local models (1) can be parsed \texttt{.ode} files. 
    They can also be directly input to the CLUE module via \texttt{sympy}. 
    All models are then parsed (2) into instances of the class \texttt{FODESystem} that supports reductions and simulations. 
    To carry out these simulations the CLUE module uses the Linear Algebra module.
}
\label{fig:clue_arch}
\end{figure}
